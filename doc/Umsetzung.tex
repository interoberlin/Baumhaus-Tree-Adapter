
\section{Umsetzung}
\subsection{Pegelwandlung}
Der 3,3V-Logikpegel wird auf 5V gewandelt
mithilfe des
\begin{center}
	Texas Instruments SN74LVC1G125DBVR\footnote{\url{http://www.ti.com/lit/ds/symlink/sn74lvc1g125.pdf}}
\end{center}
Dieser IC ben\"otigt
zum ordnungsgem\"a{\ss}en Betrieb
einen niederinduktiv platzierten St\"utzkondensator
mit etwa 10-100nF
auf dem 5V-Pegel.

\subsection{Mikrocontroller-Sockel mit Debugger-Buchse}
Zum Einsatz kommen,
wie gehabt\footnote{\url{https://github.com/interoberlin/nRF51-Boards/}},
zwei Buchsenleisten mit jeweils 2x9 Pins im Rasterma{\ss} 2mm
f\"ur die Mikrocontroller-Platine,
sowie eine 2x5 Pin IDC-Buchse
mit AVR-Debugger-Pinbelegung.

\subsection{Kupferfreie Fl\"ache unter der Antenne}
Der Bereich auf dem Adapter-PCB,
\"uber dem auf der Mikrocontroller-Platine die Antenne zu liegen kommt,
muss auf beiden Seiten von Kupfer befreit werden,
um nicht die Abstimmung der RF-Anordnung zu beeinflussen.

\subsection{Standby-Power}
Ein 3,3V-LDO versorgt den Mikrocontroller
aus der 5V-Standby-Versorgung des Netzteils.
Sobald das Netzteil angeschaltet ist,
wird der Mikrocontroller
aus dessen 3,3V-Ausgang gespei{\ss}t.

\subsection{Netzteil-Schalter}
Der N-FET 2N7002\footnote{\url{https://www.nxp.com/documents/data_sheet/2N7002.pdf}}
verbindet die Einschaltleitung des ATX-Netzteils
mit Masse.
Sein Gate
ist mit einem Ausgang
des Mikrocontrollers
verbunden
und mit einem $20k\Omega$-Pull-Down-Widerstand versehen
zur Abschaltung des Netzteils
im Falle einer Fehlfunktion.

Ein $20k\Omega$-Pull-Up-Widerstand auf der Einschaltleitung
stellt sicher, dass ohne Gate-Signal am FET
kein Einschaltsignal gegeben wird.

\subsection{Stromaufnahmemessung}
Sowohl auf der 5V-,
als auch auf der 12V-Stromversorgung
f\"ur die LEDs
wird zur Laufzeit kontinuierlich
der Stromfluss gemessen.

\begin{center}
Allegro Microsystems ACS712-30A\footnote{\url{http://www.allegromicro.com/~/media/files/datasheets/acs712-datasheet.ashx}}
\end{center}
Das resultierende Messsignal liegt
in analoger Form
an jeweils einem Pin des Mikrocontrollers
zur Auswertung
mithilfe des nRF51822-internen A/D-Wandlers
an.

\subsection{Leistungsschalter}
Jeweils ein N-FET schaltet die R\"uckf\"uhrung (Minuspole)
der 5V- und 12V-Anschl\"usse zur Netzteil-Masse durch.
Die Gates sind mit Ausg\"angen des Mikrocontrollers verbunden
und mit einem $20k\Omega$-Pull-Down-Widerstand versehen
zum Sperren der FETs
im Falle einer Fehlfunktion.

Das Package ist
aus thermischen Gr\"unden
so gew\"ahlt,
dass der FET auf der Oberseite des PCB
zu liegen kommt.

\subsection{Temperaturmessung}
Die Temperatur des Adapters wird an zwei Stellen verfolgt.
Zum Einen verf\"ugt der nRF51822 \"uber einen internen Temperatursensor.
Zum Anderen ist auf dem Adapter ein
\begin{center}
	Dallas Semiconductor DS18B20\footnote{\url{http://datasheets.maximintegrated.com/en/ds/DS18B20.pdf}}
\end{center}
verbaut,
dessen Ein-/Ausgang
mit dem Mikrocontroller verbunden ist.

\subsection{\"Uberstromschutzeinrichtung}
Unabh\"angig von einer m\"oglichen, kontrollierten Abschaltung durch den Mikrocontroller
verhindern zwei Schmelzsicherungen eine \"uberm\"a{\ss}ige Stromaufnahme.
Zum Einsatz kommen KFZ-Flachstecksicherungen mit einem Schmelzstrom zwischen 15A und 30A,
jeweils eine Sicherung an der 5V- und der 12V-Versorgung der LEDs.

Als Sockel k\"onnen KFZ-Sicherungshalter oder Flachsteckh\"ulsen zum Einsatz kommen.

Werden an die 12V-Versorgung ebenfalls LEDs angeschlossen,
z.B. \"uber ein 12V/5V-DC/DC-Wandler,
sollten zwei 15A-Sicherung verwendet werden,
da sich die Stromaufnahme der LEDs
dann auf beide Spannungen aufteilt.

Wird die 12V-Versorgung nicht verwendet,
sollte eine 25A- oder 30A-Sicherung
f\"ur die 5V-Versorgung verwendet werden.
