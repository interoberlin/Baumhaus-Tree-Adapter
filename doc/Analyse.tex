
\section{Analyse}

\subsection{Stromaufnahme des Mikrocontrollers}
Die Stromaufnahme des Mikrocontrollers liegt typischerweise
im Bereich einiger Hundert uA.
Laut der Produktspezifikation\footnote{\url{http://infocenter.nordicsemi.com/pdf/nRF51822_PS_v3.1.pdf}} kann sie bei RF-Aktivit\"at
punktuell auf 13mA ansteigen.
Man darf also guten Gewissens davon ausgehen,
dass 20mA im regul\"aren Betrieb niemals \"uberschritten werden.
Bei einer Versorgungsspannung von 3,3V
ergibt sich somit eine maximale Leistungsaufnahme von

\begin{center}
	$ 3,3V \cdot 20mA = 0,066W $
\end{center}

Bei dieser Leistungsaufnahme ist nicht mit einer
nenneswerten Erw\"armung der Mikrocontroller-Platine zu rechnen.

\subsection{Stromaufnahme der LEDs}
\label{Stromaufnahme-LEDs}
Im Versuch\footnote{\url{https://redmine.interoberlin.de/ticket/1046}}
haben wir an einem SK6812-WWA-Streifen mit 5 LEDs
an 5V Versorgungsspannung
bei voller Helligkeit
eine Stromaufnahme von 192mA
gemessen.
Daraus ergibt sich eine maximale Stromaufnahme pro LED von aufgerundet
\begin{center}
	$ \frac{192mA}{5} = 39mA $
\end{center}


Auf alle vier Streifen hochgerechnet ergibt sich eine
zu erwartende maximale Stromaufnahme von aufgerundet

\begin{center}
	$ 4 \cdot 2,5m \cdot 60 \frac{LEDs}{m} \cdot 39mA = 24A $
\end{center}

Dies entspricht einer maximalen Leistungsaufnahme pro S\"aulenseite von

\begin{center}
	$ 5V \cdot 24A = 120W $
\end{center}

\subsection{Temperaturbest\"andigkeit der Aufputzdosen}
Die Aufputzdosen bestehen laut Angaben des Herstellers
aus selbstl\"oschendem Polyvinylchlorid
mit einer nominalen Temperaturbest\"andigkeit bis 60\si{\degreeCelsius}
sowie einer Flammenwidrigkeit bis 850\si{\degreeCelsius}.

\subsection{Abmessungen der Aufputzdosen}

TODO

\subsection{Pegelwandlung mit Open Drain ist zu langsam}
Im Versuch\footnote{\url{https://redmine.interoberlin.de/ticket/1052}}
hat sich gezeigt,
dass eine (den Logik-Pegel invertierende) Aufw\"artswandlung
mithilfe einer Open Drain-Schaltung
mit dem FET
2N7000\footnote{\url{https://www.onsemi.com/pub/Collateral/2N7000-D.PDF}}
f\"ur die Ansteuerung der LEDs nicht in Frage kommt,
da der Pegel zu langsam wieder ansteigt.
Es w\"are ein Pull-Up-Widerstand von unter 380$\Omega$ erforderlich,
der im Dauerbetrieb zu Verlusten und
unerw\"unschter Erw\"armung f\"uhren w\"urde.

\subsection{ATX-Netzteil-Ausg\"ange m\"ussen belastet werden}
ATX-Netzteile eignen sich grunds\"atzlich sehr gut
f\"ur die Verwendung in eigenen L\"osungen.
F\"ur den vorliegenden Einsatzzweck bieten sie sich u.a. deshalb an,
weil sie die geregelten 5V
f\"ur die LEDs
in der erforderlichen Leistung
herausf\"uhren.
Allerdings ist zu ber\"ucksichtigen,
dass ATX-Netzteile zumeist Schaltnetzteile sind,
deren Schaltverhalten leistungsabh\"angig geregelt wird:
Abhängig von der auf der Sekund\"arseite entnommenen Leistung
wird der eingebaute Transformator betaktet,
der die 5V und 12V am Ausgang erzeugt.
In der Regel wird nur eine dieser beiden Ausgangsspannungen
im Netzteil gemessen und f\"ur die Regelung
herangezogen\footnote{\url{https://www.technic3d.com/Forum/netzteile/netzeil-extern-benutzen-und-5v-belasten-fur-die-volle-leistung-an-12v/}},
Auch kann der Tastgrad der Betaktung nicht beliebig klein
werden\footnote{\url{http://www.mikrocontroller.net/topic/215686}}.

Es sollten also beide Ausgangsspannungen belastet werden,
um sicherzustellen,
dass das Netzteil die erforderliche Leistung einspeist
und die Ausgangsspannung angemessen zu regeln imstande ist.

